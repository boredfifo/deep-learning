\documentclass[conference]{IEEEtran}
\usepackage{graphicx} % Required for inserting images
\usepackage{cite}
\usepackage{tikz}
\usepackage{listings}
\usepackage[backend=biber,style=ieee]{biblatex}
\addbibresource{references.bib}
\usetikzlibrary{positioning}
\usepackage{float}
\usepackage{stfloats}
\usepackage{amsmath}



\title{Predicting Task Failures in Autonomous Vehicle Operations}
\author{
    \IEEEauthorblockN{Mofifoluwa Ipadeola Akinwande}
    \IEEEauthorblockA{
         \textit{Electronic Engineering} \\
        \textit{Hochschule Hamm-Lippstadt}\\
        Lippstadt, Germany \\
        mofifoluwa-ipadeola.akinwande@stud.hshl.de
    }
}
\date{April 2025}
\begin{document}
\maketitle
\begin{abstract}
   Failures in autonomous vehicle(AV) operations present critical safety and reliability challenges and can be catastrophic. It is therefore important to guarantee the safety of not only the driver in the car, but also those in its environment. One way to induce safe driving is by predicting imminent task failures, allowing the AV to perform a minimum risk maneuver or yielding control to the human driver in a timely manner. Established methods like introspection and uncertainty measures help tackle this challenge. This paper explores a data-driven approach using temporal data (e.g, speed, steering angles) from successful and failed autonomous driving operations to implement a small scale Long Short-Term Memory(LSTM) network that helps predict based on data from up to 5 seconds before whether a failure is imminent or not.
\end{abstract}
\section{Introduction}
The autonomous vehicle(AV) industry is constantly expanding, fueled by significant financial investments into multiple research facets with the major goal of achieving fully automated driving. Despite this progress, most autonomous solutions today still operate under restricted conditions and require a human driver to be present for supervision.  Critically, "failures" (failures is quoted because in this context there is no general definition of failure as it may depend on a specific driving model or use case) still occur and can be catastrophic. Common cases of when "failures" occur include when a model encounters a scene not present in its training data or when a particular scenario is too difficult to navigate. When such failures occur control of the vehicle is given back to the human driver. 
Various research methods exist to mitigate these risks, including methods like uncertainty measures, where the model outputs a confidence score of whether it finds a particular scene difficult or not\cite{hecker2018}. Dependent on the score it can continue with its automation or give control to the human. Other methods \cite{kuhn2020} treat the car as black box and employ introspection whereby based on historical failed sequences, a secondary model is trained to predict whether a failure is imminent.
Focusing on the latter method, this paper proposes a data-driven methodology. Utilizing data from simulated driving scenarios, a long short term memory(LSTM) network is modeled that outputs predictions of whether a failure is imminent or not every second based on input of past sequences. The paper structure is as follows: section 2 provides a background into failure prediction in autonomous vehicles and the current solutions, section 3 highlights the small scale implementation of the LSTM netweork and the results before concluding in section 4.
\section{Background}
\section{Implementation}
\section{Conclusion}
\printbibliography
\end{document}
