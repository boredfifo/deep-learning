\documentclass[conference]{IEEEtran}
\usepackage{graphicx} % Required for inserting images
\usepackage{cite}
\usepackage{tikz}
\usepackage{listings}
\usepackage[backend=biber,style=ieee]{biblatex}
\addbibresource{references.bib}
\usetikzlibrary{positioning}
\usepackage{float}
\usepackage{stfloats}
\usepackage{amsmath}



\title{Predicting Task Failures in Autonomous Vehicle Operations}
\author{
    \IEEEauthorblockN{Mofifoluwa Ipadeola Akinwande}
    \IEEEauthorblockA{
         \textit{Electronic Engineering} \\
        \textit{Hochschule Hamm-Lippstadt}\\
        Lippstadt, Germany \\
        mofifoluwa-ipadeola.akinwande@stud.hshl.de
    }
}
\date{April 2025}
\begin{document}
\maketitle
\begin{abstract}
   Failures in autonomous vehicle(AV) operations present critical safety and reliability challenges and can be catastrophic. It is therefore important to guarantee the safety of not only the driver in the car, but also those in its environment. One way to induce safe driving is by predicting imminent task failures, allowing the AV to perform a minimum risk maneuver or yielding control to the human driver in a timely manner. Established methods like introspection and uncertainty measures help tackle this challenge. This paper explores a data-driven approach using temporal data (e.g, speed, steering angles) from successful and failed autonomous driving operations to implement a small scale Long Short-Term Memory(LSTM) network that helps predict based on data from up to 5 seconds before whether a failure is imminent or not.
\end{abstract}
\section{Introduction}
The autonomous vehicle(AV) industry is constantly expanding, fueled by significant financial investments into multiple research facets with the major goal of achieving fully automated driving. Despite this progress, most autonomous solutions today still operate under restricted conditions and require a human driver to be present for supervision.  Critically, "failures" (failures is quoted because in this context there is no general definition of failure as it may depend on a specific driving model or use case) still occur and can be catastrophic. 
Most autonomous vehicles consist of three main modules responsible for their overall functioning; the \textbf{perception}, \textbf{planning} and \textbf{control} module. The perception module takes in a multitude of inputs from a variety of sensors, typically cameras, lidar, and other low-level sensors, and transforms these inputs to a suitable environmental representation with the aid of subsystems designed for \textbf{object detection} or \textbf{semantic segementation}. The planning module and its subsystems then utilize these outputs to handle route, behavior, and motion planning of the vehicle, determining how the vehicle should navigate its environment. Finally, the control module is responsible for the implementation of the outputs (typically on a hardware level) of the planning module to execute the chosen maneuvers.\\
Failure in any of these components can result in an overall system-level failure, potentially compromising safety. As a result, it is important to researchers and industry practitioners to be able to reduce failures as much as possible in these components to ensure safe automated driving. One possible approach is to predict failures in these components before they occur. That way automation is yielded back to a human driver or a safe maneuver is executed in a timely manner.\\
Introspection, where a model learns from historical data of both failed and successful scenarios to infer the likelihood that a particular input may lead to failure, is an interesting method for such predictive analysis, and much research and implementation has been done in this respect\cite{kuhn2020, kuhn2022}.\\
Focusing on this method, this paper proposes a data-driven methodology. Utilizing data from simulated driving scenarios, a long short term memory(LSTM) network is modeled that outputs predictions of whether a failure is imminent or not every second based on input of past sequences. The paper structure is as follows: section 2 provides a background into failure prediction in autonomous vehicles and the current solutions, section 3 highlights the small scale implementation of the LSTM network and the results before concluding in section 4.
\section{Background}
\section{Implementation}
\section{Conclusion}
\printbibliography
\end{document}
